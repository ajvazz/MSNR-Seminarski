% !TEX encoding = UTF-8 Unicode
\documentclass[a4paper]{article}

\usepackage{color}
\usepackage{url}
\usepackage[T2A]{fontenc} % enable Cyrillic fonts
\usepackage[utf8]{inputenc} % make weird characters work
\usepackage{graphicx}

\usepackage[english,serbian]{babel}
%\usepackage[english,serbianc]{babel} %ukljuciti babel sa ovim opcijama, umesto gornjim, ukoliko se koristi cirilica

\usepackage[unicode]{hyperref}
\hypersetup{colorlinks,citecolor=green,filecolor=green,linkcolor=blue,urlcolor=blue}

\usepackage{listings}

%\newtheorem{primer}{Пример}[section] %ćirilični primer
\newtheorem{primer}{Primer}[section]

\definecolor{mygreen}{rgb}{0,0.6,0}
\definecolor{mygray}{rgb}{0.5,0.5,0.5}
\definecolor{mymauve}{rgb}{0.58,0,0.82}

\lstset{ 
  backgroundcolor=\color{white},   % choose the background color; you must add \usepackage{color} or \usepackage{xcolor}; should come as last argument
  basicstyle=\scriptsize\ttfamily,        % the size of the fonts that are used for the code
  breakatwhitespace=false,         % sets if automatic breaks should only happen at whitespace
  breaklines=true,                 % sets automatic line breaking
  captionpos=b,                    % sets the caption-position to bottom
  commentstyle=\color{mygreen},    % comment style
  deletekeywords={...},            % if you want to delete keywords from the given language
  escapeinside={\%*}{*)},          % if you want to add LaTeX within your code
  extendedchars=true,              % lets you use non-ASCII characters; for 8-bits encodings only, does not work with UTF-8
  firstnumber=1000,                % start line enumeration with line 1000
  frame=single,	                   % adds a frame around the code
  keepspaces=true,                 % keeps spaces in text, useful for keeping indentation of code (possibly needs columns=flexible)
  keywordstyle=\color{blue},       % keyword style
  language=Python,                 % the language of the code
  morekeywords={*,...},            % if you want to add more keywords to the set
  numbers=left,                    % where to put the line-numbers; possible values are (none, left, right)
  numbersep=5pt,                   % how far the line-numbers are from the code
  numberstyle=\tiny\color{mygray}, % the style that is used for the line-numbers
  rulecolor=\color{black},         % if not set, the frame-color may be changed on line-breaks within not-black text (e.g. comments (green here))
  showspaces=false,                % show spaces everywhere adding particular underscores; it overrides 'showstringspaces'
  showstringspaces=false,          % underline spaces within strings only
  showtabs=false,                  % show tabs within strings adding particular underscores
  stepnumber=2,                    % the step between two line-numbers. If it's 1, each line will be numbered
  stringstyle=\color{mymauve},     % string literal style
  tabsize=2,	                   % sets default tabsize to 2 spaces
  title=\lstname                   % show the filename of files included with \lstinputlisting; also try caption instead of title
}

\begin{document}

\title{Pouzdanost softvera\\ \small{Seminarski rad u okviru kursa\\Metodologija stručnog i naučnog rada\\ Matematički fakultet}}

\author{Nenad Ajvaz, Stefan Kapunac, Filip Jovanović, Aleksandra Radosavljević\\ nenadajvaz@hotmail.com, stefankapunac@gmail.com, \\jovanovic16942@gmail.com, aleksandraradosavljevic.@live.com}

%\date{9.~april 2015.}

\maketitle

\abstract{
ABSTRAKT ABSTRAKT ABSTRAKT ABSTRAKT ABSTRAKT ABSTRAKT ABSTRAKT ABSTRAKT
ABSTRAKT ABSTRAKT ABSTRAKT ABSTRAKT ABSTRAKT ABSTRAKT ABSTRAKT ABSTRAKT
ABSTRAKT ABSTRAKT ABSTRAKT ABSTRAKT ABSTRAKT ABSTRAKT ABSTRAKT ABSTRAKT
ABSTRAKT ABSTRAKT ABSTRAKT ABSTRAKT ABSTRAKT ABSTRAKT ABSTRAKT ABSTRAKT
ABSTRAKT ABSTRAKT ABSTRAKT ABSTRAKT ABSTRAKT ABSTRAKT ABSTRAKT ABSTRAKT
ABSTRAKT ABSTRAKT ABSTRAKT ABSTRAKT ABSTRAKT ABSTRAKT ABSTRAKT ABSTRAKT
ABSTRAKT ABSTRAKT ABSTRAKT ABSTRAKT ABSTRAKT ABSTRAKT ABSTRAKT ABSTRAKT
}

\tableofcontents

\newpage

\section{Uvod}
\label{sec:uvod}

- Velika uloga softvera u danasnje vreme \\
- Softver je sve rasprostranjeniji, uskoro ce postati osnovna radna snaga \\
- Sa povecanjem uloge softvera u drustvu, njegova pouzdanost ima veci znacaj, jer
od softvera vec uveliko zavise i ljudski zivoti \\
- Za razliku od ljudi, softver ne pravi slucajne greske, ali do gresaka \\
ipak dolazi iz raznih razloga (mali promaci i male greske se vremenom ispoljavaju) \\
- U nastavku cemo predstaviti mere i modele pouzdanosti softvera, metode za poboljsanje pouzdanosti, kao i primere i u kojima su greske u sistemu dovele do ozbiljnih problema


\section{Primeri padova softvera}
\label{sec:primeri}
- Izmedju ostalih, dodati primer za Boing (sad ovaj sto je pao, 10. marta) \\

\section{Verifikacija softvera}	
\label{sec:verifikacija}

- Bice okaceno

\section{Modeli i metrike pouzdanosti softvera}	
\label{sec:modeli_metrike}

- Veoma je vazno proceniti broj potencijalnih gresaka u sistemu \\
- Istorijski gledano, na svakih 1000 naredbi, desi se u proseku 8 gresaka \\
- Postoje deterministicki i probabilisticki modeli kojima se moze (pr)oceniti trenutni broj gresaka u sistemu \\
- Deterministicki meri broj instrukcija, operatora, operanada i ostalih tehnickih detalja, ali i broj gresaka \\
- Deterministicki se ne oslanja na slucajne dogadjaje, sve je egzaktno
- Probabilisticki predstavlja ispoljavanja gresaka, ali i njihovo uklanjanje, kao probabilisticke dogadjaje \\
- U nastavku poglavlja, nabrojacemo glavne predstavnike oba ova modela

\subsection{Deterministicki modeli}
\label{sec:deterministicki}

- dva glavna su: Holstedova metrika i Mek-Kejbova ciklomaticna slozenost
- TODO

\subsection{Probabilisticki modeli}
\label{sec:probabilisticki}

- Odabrati nekoliko i napisati po recenicu o svakom \\
- TODO \\

\section{Budućnost softvera}
\label{buducnost}

- TODO \\

\section{Zaključak}
\label{sec:zakljucak}

- TODO \\

\addcontentsline{toc}{section}{Literatura}
\appendix
\bibliography{seminarski} 
\bibliographystyle{plain}

\appendix
\section{Dodatak}
Ovde pišem dodatne stvari, ukoliko za time ima potrebe.
Ovde pišem dodatne stvari, ukoliko za time ima potrebe.
Ovde pišem dodatne stvari, ukoliko za time ima potrebe.
Ovde pišem dodatne stvari, ukoliko za time ima potrebe.
Ovde pišem dodatne stvari, ukoliko za time ima potrebe.


\end{document}
